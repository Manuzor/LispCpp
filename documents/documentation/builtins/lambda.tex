\subsection{lambda}
\label{builtins/lambda}

\subsubsection*{Declaration}
\begin{lstlisting}
(lambda ([arg...]) code [code...])
\end{lstlisting}

\subsubsection*{Parameters}
\begin{description}
	\item[arg...] Zero or more unevaluated symbols to bind values to when the given lambda
	\item[code/code...] An expression that will be evaluated when the produced lambda object is called. These code expressions are executed in the order they are given.
\end{description}

\subsubsection*{Description}
Creates a procedure object that can be used to execute user-defined code. Will return the evaluated result of the last code argument when executed.

A lambda needs at least one code element in its body.

\subsubsection{Examples}
\begin{lstlisting}
; Returns the integer 42 when executed.
(lambda() 42)

; Binds the value 42 to the symbol x.
(define x ((lambda() 42)))

; Short-hand syntax to create lambda objects.
(define (make-adder x) (lambda (n) (+ x n)))
\end{lstlisting}
