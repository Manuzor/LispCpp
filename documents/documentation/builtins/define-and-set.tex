\subsection{Syntax 'define' and 'set!'}
\label{builtins/define}

\subsubsection*{Declaration}
\begin{lstlisting}
(define symbol expression)
(define (symbol [arg]) code [code...])
(set! symbol expression)
(set! (symbol [arg]) code [code...])
\end{lstlisting}

\subsubsection*{Parameters}
\begin{description}
	\item[symbol] The unevaluated symbol to bind the result of the expression to.
	\item[expression] The result of expression is bound to the specified symbol.
	\item[code] An expression that will be evaluated when the lambda is called.
	\item[arg] An optional unevaluated symbol to bind values to that are available when executing all code argument.
\end{description}

\subsubsection*{Description}
Binds the value of the given evaluated expression to the given unevaluated symbol.

The only difference between \lstinline|define| and \lstinline|set!| is basically that \lstinline|set!| can only overwrite existing bindings. If a binding for a symbol does not exist yet, \lstinline|set!| will fail.

The second forms are equivalent to the following:
\begin{lstlisting}
(define symbol (lambda ([arg...]) code [code...]))
\end{lstlisting}

\subsubsection*{Examples}
\begin{lstlisting}
(define x 1)
(define value "value")
(define another-value (+ 1 2 3)) ; another-value is then equal to 6
(define (make-adder x) (lambda (n) (+ x n)))
\end{lstlisting}
