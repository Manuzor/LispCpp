\subsection{Procedure '+', '-', '*', '/', and '\%'}
\label{builtins/arithmetic-operations}

\subsubsection*{Declaration}
\begin{lstlisting}
(+ [op...])
(- op [op...])
(* [op...])
(/ op [op...])
(% op op)
\end{lstlisting}

\subsubsection*{Parameters}
\begin{description}
	\item[op] An operand of the arithmetic operation. Must be of either integer or a float type. Note that the procedure '\%' only accepts integer type operands.
\end{description}

\subsubsection*{Description}
These are the typical arithmetic operations one would expect. The procedures \lstinline|+| and \lstinline|*| return 0 and 1 when no argument is given, respectively. The procedure \lstinline|%| requires exactly 2 arguments of type integer, all other arithmetic operations take either integer or float type numbers, or a mix of them.

\subsubsection*{Examples}
\begin{lstlisting}
(+ 1 2) ; Yields 3
(+ -1 2) ; Yields 1
(- 44 2) ; Yields 42
(*) ; Yields 1
(* 7 6) ; Yields 42
(/ 1 2) ; Yields 0.5
(% 29 12) ; Yields 5
\end{lstlisting}
