\subsection{Syntax 'and' and 'or'}
\label{builtins/and-or}

\subsubsection*{Declaration}
\begin{lstlisting}
(and [expression...])
(or [expression...])
\end{lstlisting}

\subsubsection*{Parameters}
\begin{description}
	\item[expression...] An expression to be evaluated. Note that all expressions are evaluated lazily.
\end{description}

\subsubsection*{Description}
Syntax \lstinline|and| evaluates the given expressions in the order they are given and returns \true{} if \textbf{all} evaluate to be not \false{}. The first expression that is evaluated to be \false{} will cancel the evaluation of the remaining expressions. If no argument is given, \true{} is returned.

Syntax \lstinline|or| evaluates the given expressions in the order they are given and returns \true if \textbf{any} evaluate to be \true{}. The first expression that is evaluated to be \true{} will cancel the evaluation of the remaining expressions. If no argument is given, \false{} is returned.

\subsubsection*{Examples}
\begin{lstlisting}
(and )
; Both examples below will always print "hello"
(and (print "hello") #f (print "world"))
(or  #f (print "hello") (print "world"))
\end{lstlisting}
