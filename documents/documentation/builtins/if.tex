\subsection{Syntax 'if'}
\label{builtins/if}

\subsubsection*{Declaration}
\begin{lstlisting}
(if condition then else)
\end{lstlisting}

\subsubsection*{Parameters}
\begin{description}
	\item[condition] The condition to be evaluated.
	\item[then] Will be evaluated if the given condition is not \lstinline|#f|.
	\item[else] Will be evaluated if the given condition is \lstinline|#f|.
\end{description}

\subsubsection*{Description}
If the given \lstinline|condition| evaluates to something that is not \lstinline|#f|, the \lstinline|then| argument is evaluated and the result of that is returned. In case the given \lstinline|condition| is \lstinline|#f|, the \lstinline|else| argument is evaluated and returned.

Note that it is guaranteed that either \lstinline|then| or \lstinline|else| is evaluated, but never both or none.

\subsubsection*{Examples}
\begin{lstlisting}
(if #t 1 2) ; Yields 1
(if #f 1 2) ; Yields 2
(if #f (print "Hello") (print "Goodbye")) ; prints "Goodbye"
\end{lstlisting}
