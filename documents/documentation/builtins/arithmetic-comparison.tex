\subsection{Procedure '>', '>=', '=', '<', and '<='}
\label{builtins/arithmetic-comparison}

\subsubsection*{Declaration}
\begin{lstlisting}
(>  op op [op...])
(>= op op [op...])
(=  op op [op...])
(<  op op [op...])
(<= op op [op...])
\end{lstlisting}

\subsubsection*{Parameters}
\begin{description}
	\item[op] The operand of either integer or float type to compare against one or more other operands.
\end{description}

\subsubsection*{Description}
These are the typical arithmetic comparison operators. Each of them takes at least 2 arguments of either integer or float type.

\subsubsection*{Examples}
\begin{lstlisting}
(<  1 2) ; #t
(<= 1 2) ; #t
(=  1 2) ; #f
(>  1 2) ; #f
(>= 1 2) ; #f
\end{lstlisting}
