\subsection{Syntax 'begin'}
\label{builtins/begin}

\subsubsection*{Declaration}
\begin{lstlisting}
(begin [expression...])
\end{lstlisting}

\subsubsection*{Parameters}
\begin{description}
	\item[expression...] One or more expressions that will be evaluated in the same order that they are given.
\end{description}

\subsubsection*{Description}
Acts basically like a \lstinline|lambda| except that it does not accept additional arguments and does not create an inner environment. The surrounding environment becomes the one within the \lstinline|begin| statement. When used in an assignment or return context, the last evaluated result within the \lstinline|begin| statement's body will be returned.

\subsubsection*{Examples}
\begin{lstlisting}
(begin (print "hello") (print "world")) ; Prints "hello\nworld\n"
(define x (begin 1 2 3)) ; x equals to 3 now.

; Will never return 1337 but print a string and return 42.
(if #f 1337 (begin (print "Returning the answer.") 42))
\end{lstlisting}
