% !TeX encoding = UTF-8
% !TeX spellcheck = en_US
% !TeX root = documentation.tex
\documentclass[a4paper]{scrartcl}
\usepackage{fontspec}
\setmonofont{Consolas}
\usepackage{color}
	\definecolor{codeComment}{RGB}{0,128,0}
	\definecolor{codeKeyword}{RGB}{0,0,255}
	\definecolor{codeIdentifier}{RGB}{64,64,64}
	\definecolor{codeString}{RGB}{163,21,21}
	\definecolor{codeLineNumbers}{RGB}{0,0,0}
\usepackage{listings}
	\lstset{language=Lisp}
	\lstset{basicstyle=\ttfamily\small}
	\lstset{keywordstyle=\color{codeKeyword}}
	\lstset{commentstyle=\color{codeComment}}
	\lstset{identifierstyle=\color{codeIdentifier}}
	\lstset{stringstyle=\color{codeString}}
	\lstset{showstringspaces=false}
	\lstset{tabsize=4}
	\lstset{captionpos=b}
	\lstset{numbers=none}
	\lstset{numberstyle=\ttfamily\color{codeLineNumbers}}
	\lstset{breaklines=true}
	\lstset{morekeywords={define,if,lambda,quote,begin}}
\usepackage{todonotes}
\usepackage{enumitem}
\setlist[description]{leftmargin=\parindent,labelindent=\parindent}
\usepackage[pdfborder={0 0 0 0}]{hyperref}

% Title Page
\title{LispCpp Documentation}
\subtitle{Course "Design and Implementation of modern Programming languages" held by Claus Gittinger}
\author{Manuel Maier, Matriculation Number 28535\\
		Stuttgart Media University}
\date{\today}

\begin{document}
\maketitle
\tableofcontents
\clearpage

\section{Features}
	\todo{Describe all features in a list with concise comments.}
	
\section{Built-ins}
	Within this section, a simple declaration syntax is used, mixed with the regular LISP syntax:

	Declarations for procedures look like they're called at that instance. This was done to keep it more simple.

	Parameters preceded by an ellipse describe that the parameter may appear zero, one, or multiple times: \lstinline|param...|

	Parameters enclosed in brackets are optional: \lstinline|[param]| or \lstinline|[param...]|

	Everything else is regular LISP syntax. Note that these are not the actual declarations, they're just there for the purposes of this documentation. They are built-ins after all.

	\subsection{Pre-defined symbols}
\label{builtins/symbols}

\subsubsection*{Symbols}
\begin{lstlisting}
#t #f #v null
\end{lstlisting}

\subsubsection*{Description}
The symbol \lstinline|#t| represents the value 'true'.\\
The symbol \lstinline|#f| represents the value 'false'.\\
The symbol \lstinline|#v| represents the non-value, i.e. "void".\\
The symbol \lstinline|null| represents the empty list.

\subsubsection*{Examples}
\begin{lstlisting}
(if #t 1 2) ; Will always yield 1
(if #f 1 2) ; Will always yield 2
(print #v) ; Will never print anything
(cons 1 (cons 2 null)) ; Creates a regular list equivalent to '(1 2)
\end{lstlisting}


	\subsection{Syntax 'define' and 'set!'}
\label{builtins/define}

\subsubsection*{Declaration}
\begin{lstlisting}
(define symbol expression)
(define (symbol [arg]) code [code...])
(set! symbol expression)
(set! (symbol [arg]) code [code...])
\end{lstlisting}

\subsubsection*{Parameters}
\begin{description}
	\item[symbol] The unevaluated symbol to bind the result of the expression to.
	\item[expression] The result of expression is bound to the specified symbol.
	\item[code] An expression that will be evaluated when the lambda is called.
	\item[arg] An optional unevaluated symbol to bind values to that are available when executing all code argument.
\end{description}

\subsubsection*{Description}
Binds the value of the given evaluated expression to the given unevaluated symbol.

The only difference between \lstinline|define| and \lstinline|set!| is basically that \lstinline|set!| can only overwrite existing bindings. If a binding for a symbol does not exist yet, \lstinline|set!| will fail.

The second forms are equivalent to the following:
\begin{lstlisting}
(define symbol (lambda ([arg...]) code [code...]))
\end{lstlisting}

\subsubsection*{Examples}
\begin{lstlisting}
(define x 1)
(define value "value")
(define another-value (+ 1 2 3)) ; another-value is then equal to 6
(define (make-adder x) (lambda (n) (+ x n)))
\end{lstlisting}

	\subsection{Syntax lambda}
\label{builtins/lambda}

\subsubsection*{Declaration}
\begin{lstlisting}
(lambda ([arg...]) code [code...])
\end{lstlisting}

\subsubsection*{Parameters}
\begin{description}
	\item[arg...] Zero or more unevaluated symbols to bind values to when the given lambda
	\item[code/code...] An expression that will be evaluated when the produced lambda object is called. These code expressions are executed in the order they are given.
\end{description}

\subsubsection*{Description}
Creates a procedure object that can be used to execute user-defined code. Will return the evaluated result of the last code argument when executed.

A lambda needs at least one code element in its body.

\subsubsection*{Examples}
\begin{lstlisting}
; Returns the integer 42 when executed.
(lambda () 42)

; Binds the value 42 to the symbol x.
(define x ((lambda () 42)))

; Short-hand syntax to create lambda objects.
(define (make-adder x) (lambda (n) (+ x n)))
\end{lstlisting}

	\subsection{Syntax 'quote'}
\label{builtins/quote}

\subsubsection*{Declaration}
\begin{lstlisting}
(quote expression)
\end{lstlisting}

\subsubsection*{Parameters}
\begin{description}
	\item[expression] The expression is not evaluated but returned as the symbol or list that it represents.
\end{description}

\subsubsection*{Description}
Quotes the given expression argument. Note that only one argument can be specified.

\subsubsection*{Examples}
\begin{lstlisting}
(quote hello) ; Yields the symbol 'hello', not the value that might be bound to that symbol
(quote (1 2 3)) ; Logically equivalent to (list 1 2 3)
\end{lstlisting}

	\subsection{Syntax 'begin'}
\label{builtins/begin}

\subsubsection*{Declaration}
\begin{lstlisting}
(begin [expression...])
\end{lstlisting}

\subsubsection*{Parameters}
\begin{description}
	\item[expression...] One or more expressions that will be evaluated in the same order that they are given.
\end{description}

\subsubsection*{Description}
Acts basically like a \lstinline|lambda| except that it does not accept additional arguments and does not create an inner environment. The surrounding environment becomes the one within the \lstinline|begin| statement. When used in an assignment or return context, the last evaluated result within the \lstinline|begin| statement's body will be returned.

\subsubsection*{Examples}
\begin{lstlisting}
(begin (print "hello") (print "world")) ; Prints "hello\nworld\n"
(define x (begin 1 2 3)) ; x equals to 3 now.

; Will never return 1337 but print a string and return 42.
(if #f 1337 (begin (print "Returning the answer.") 42))
\end{lstlisting}

	\subsection{Syntax 'if'}
\label{builtins/if}

\subsubsection*{Declaration}
\begin{lstlisting}
(if condition then else)
\end{lstlisting}

\subsubsection*{Parameters}
\begin{description}
	\item[condition] The condition to be evaluated.
	\item[then] Will be evaluated if the given condition is not \lstinline|#f|.
	\item[else] Will be evaluated if the given condition is \lstinline|#f|.
\end{description}

\subsubsection*{Description}
If the given \lstinline|condition| evaluates to something that is not \lstinline|#f|, the \lstinline|then| argument is evaluated and the result of that is returned. In case the given \lstinline|condition| is \lstinline|#f|, the \lstinline|else| argument is evaluated and returned.

Note that it is guaranteed that either \lstinline|then| or \lstinline|else| is evaluated, but never both or none.

\subsubsection*{Examples}
\begin{lstlisting}
(if #t 1 2) ; Yields 1
(if #f 1 2) ; Yields 2
(if #f (print "Hello") (print "Goodbye")) ; prints "Goodbye"
\end{lstlisting}


\end{document}
