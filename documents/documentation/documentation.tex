% !TeX encoding = UTF-8
% !TeX spellcheck = en_US
% !TeX root = documentation.tex
\documentclass[a4paper]{scrartcl}
\usepackage{fontspec}
\setmonofont{Consolas}
\usepackage{color}
	\definecolor{codeComment}{RGB}{0,128,0}
	\definecolor{codeKeyword}{RGB}{0,0,255}
	\definecolor{codeIdentifier}{RGB}{64,64,64}
	\definecolor{codeString}{RGB}{163,21,21}
	\definecolor{codeLineNumbers}{RGB}{0,0,0}
\usepackage{listings}
	\lstset{language=Lisp}
	\lstset{basicstyle=\ttfamily\small}
	\lstset{keywordstyle=\color{codeKeyword}}
	\lstset{commentstyle=\color{codeComment}}
	\lstset{identifierstyle=\color{codeIdentifier}}
	\lstset{stringstyle=\color{codeString}}
	\lstset{showstringspaces=false}
	\lstset{tabsize=4}
	\lstset{captionpos=b}
	\lstset{numbers=none}
	\lstset{numberstyle=\ttfamily\color{codeLineNumbers}}
	\lstset{breaklines=true}
	\lstset{morekeywords={define,if,lambda,quote,begin}}
\usepackage{todonotes}
\usepackage{enumitem}
\setlist[description]{leftmargin=\parindent,labelindent=\parindent}
\usepackage[pdfborder={0 0 0 0}]{hyperref}

\newcommand{\false}{\lstinline|\#f|}
\newcommand{\true}{\lstinline|\#t|}

% Title Page
\title{LispCpp\\
	Documentation and Reference Manual}
\subtitle{"Design and Implementation of Modern Programming languages"\\
	Summer Term 2014\\
	Held by Claus Gittinger}
\author{Manuel Maier, Matriculation Number 28535\\
		Stuttgart Media University}
\date{\today}

\begin{document}
\maketitle
\tableofcontents
\clearpage

\section{Introduction and Document Overview}
\label{overview}
	This document is the documentation and reference manual for the LISP interpreter "LispCpp", written in C++ by Manuel Maier for the course "Design and Implementation of Modern Programming languages" held by Claus Gittinger at the Stuttgart Media University in the summer term 2014. It might seem like quite a long document, but do not worry, most of it is just a list of all the built-ins.

	For the guide on how to use the interpreter, check section \ref{usage}. For an overview of the implemented features, check section \ref{features}. Section \ref{builtins} lists all built-in symbols and procedures and explains how to use them.

\section{Third Party Software}
\label{thirdparty}
\subsection{ezEngine}
\label{ez}
	\url{http://ezengine.net/}\\
	Used for string operations, basic containers (dynamic array, hash map, etc.), file I\slash{}O, preprocessor utilities, and platform abstraction. The ezEngine is an open-source project licensed under a Creative Commons Attribution 3.0 Unported License.

\section{Feature Overview}
\label{features}
	This interpreter is written in C++11 and has only been tested using Microsoft Visual Studio 2013 and the compiler that ships with it. For the most part, the code is platform independent, mostly due to usage of the ezEngine (see \ref{ez}), except for the implementation of the garbage collector (see \ref{gc}), which uses Win32 specific functions.

	The interpreter uses a Baker-style garbage collection algorithm, continuation passing, and supports integers as well as floating point numbers.

\subsection{Garbage Collection}
\label{gc}
	The garbage collector can be configured to use a specific number of memory pages as initial memory size. This defaults to 1 page of 4096 KiB. For memory allocation the Win32 API function \lstinline|VirtualAlloc| is used. For debugging purposes, \lstinline|VirtualProtect| was an enormous help finding bugs in the implementation of the garbage collector.

	Since it is a Baker style garbage collector, there are two spaces of memory used, namely the eden space and the survivor space. After each collection cycle, the roles of the eden space and survivor space are swapped. outside of the collection cycle, i.e. during regular runtime, the survivor space is not used at all and is protected using \lstinline|VirtualProtect| and the \lstinline|PAGE_NOACCESS| argument, which issues a hardware error if the survivor space is accessed in any kind of way, be it reading, writing, or execution access.

	After a collection cycle, if the fill level of the new eden space is above a certain threshold\footnote{This threshold is configurable.}, a flag is set to indicate that the garbage collector needs to allocate new memory before the next collection cycle. When a new collection cycle is performed, and the flag is set, the number of currently allocated memory pages is adjusted so that it can take enough LISP objects to stay below the threshold mentioned before. The amount of memory allocated by the garbage collector does never shrink. A collection cycle can be triggered manually, but is usually triggered when the user tries to allocate memory for new LISP objects.

\subsection{Numbers}
\label{numbers}
	Numbers in LispCpp come in two flavors:
	\begin{itemize}
		\item Signed 64-bit integers
		\item IEEE 754 double precision floating point numbers.
	\end{itemize}

	Except for the modulo procedure \lstinline|%| all arithmetic operations described in \ref{builtins/arithmetic-operations} and \ref{builtins/arithmetic-comparison} work with either of these flavors as arguments.
	
	There is no support for big integers (yet).
	
\subsection{Top-Level Environments}
\label{env}
	There are two kinds of top-level environments used in the current runtime implementation:
	\begin{itemize}
		\item Syntax environment
		\item Character-Macro environment
	\end{itemize}

	The syntax environment stores all syntax procedures. It is available at both parsing time as well as evaluation time. At parsing time the syntax environment is used to replace occurrences of syntax symbols with the actual syntax objects they're bound to. For example, if the reader parses the string \lstinline|"(define x 1)"| it produces a cons-list where the first element is the built-in syntax object called 'define' (see \ref{builtins/define}).

	The syntax environment, however, is not the environment the user defines variables in. For this purpose an environment called 'global' is created when the interpreter is started. The parent environment of 'global' is the syntax environment. This allows the user to shadow built-in syntax symbols with custom implementations, if they wish to do that. The global environment is only available at evaluation-time, not at parsing-time.

	The character-macro environment stores all built-in procedure objects that would be called at parsing-time when a certain character was parsed. In the current implementation there are only two built-in character macros:
	\begin{description}
		\item[(] Used to parse a list, such as \lstinline|(1 2 3)|
		\item["] Used to parse a string object, such as "hello world"
	\end{description}

	A future feature that could be implemented using this environment is support for user-defined character macros. Such a feature would enable the user to parse an XML document as a series of LISP objects, for example.

\section{Usage}
\label{usage}
	Assume the following directory structure:
	\begin{lstlisting}
	lcpp/output/bin/WinVs2013Release64/lcppCLICont.exe
	lcpp/data1/base/init.lisp
	lcpp/data1/user/tak.lisp
	\end{lstlisting}

	The folder "lcpp/output/bin/" is where the binary executables are located at. The most important one is "lcppCLICont.exe", which is used to run the interpreter.

	The folder "lcpp/data1/base/" is the directory used for the interpreter's internals. Currently, it contains only the file "init.lisp". This file is loaded and executed when the interpreter is started, even before entering the REPL.

	The folder "lcpp/data1/user/" is the place where all user-defined scripts should be located in. This is also the directory used by the built-in procedure \lstinline|file.open| (see \ref{builtins/file.open}) to find the files it should open. For example passing the argument "hello.lisp" to \lstinline|file.open|, it will try to open the file "lcpp/data1/user/hello.lisp".

	You have to run the interpreter from the "lcpp" directory so that it can find the "data1" directory in its working directory.

\subsection{REPL}
\label{usage/repl}
	The interactive REPL\footnote{Read-Eval-Print-Loop} accepts multi-line input. This feature can be triggered by opening a parenthesis and hitting Return or Enter. The REPL basically waits for the parenthesis to be balanced. If a syntax error is produced, such as \lstinline|())|, the REPL will issue an error, informing the user of the line and column it expected something else than was given. The interpreter is also counting the number of lines the user gives as input and displays this number to the left of the prompt character such as \lstinline|128>| when 128 lines were punched in.

\section{Built-in Symbols, Syntax, and Procedures}
\label{builtins}
	This section is a compilation of all available objects and built-in procedures available to the user out of the box. A simple declaration syntax is used in this section, mixed with the regular LISP syntax:

	\begin{itemize}
		\item Parameters preceded by an ellipse describe that the parameter may appear zero, one, or multiple times: \lstinline|param...|
		\item Parameters enclosed in brackets are optional: \lstinline|[param]| or \lstinline|[param...]|
		\item Procedure declarations follow this pattern:\\
			  \lstinline|(procedure-name required [optional])|
	\end{itemize}

	Everything else is regular LISP syntax. Note that these are not the actual declarations, they're just there for the purposes of this documentation. These are built-ins after all.

	% Symbols
	\subsection{Pre-defined symbols}
\label{builtins/symbols}

\subsubsection*{Symbols}
\begin{lstlisting}
#t #f #v null
\end{lstlisting}

\subsubsection*{Description}
The symbol \lstinline|#t| represents the value 'true'.\\
The symbol \lstinline|#f| represents the value 'false'.\\
The symbol \lstinline|#v| represents the non-value, i.e. "void".\\
The symbol \lstinline|null| represents the empty list.

\subsubsection*{Examples}
\begin{lstlisting}
(if #t 1 2) ; Will always yield 1
(if #f 1 2) ; Will always yield 2
(print #v) ; Will never print anything
(cons 1 (cons 2 null)) ; Creates a regular list equivalent to '(1 2)
\end{lstlisting}


	% Syntax
	\subsection{Syntax 'define' and 'set!'}
\label{builtins/define}

\subsubsection*{Declaration}
\begin{lstlisting}
(define symbol expression)
(define (symbol [arg]) code [code...])
(set! symbol expression)
(set! (symbol [arg]) code [code...])
\end{lstlisting}

\subsubsection*{Parameters}
\begin{description}
	\item[symbol] The unevaluated symbol to bind the result of the expression to.
	\item[expression] The result of expression is bound to the specified symbol.
	\item[code] An expression that will be evaluated when the lambda is called.
	\item[arg] An optional unevaluated symbol to bind values to that are available when executing all code argument.
\end{description}

\subsubsection*{Description}
Binds the value of the given evaluated expression to the given unevaluated symbol.

The only difference between \lstinline|define| and \lstinline|set!| is basically that \lstinline|set!| can only overwrite existing bindings. If a binding for a symbol does not exist yet, \lstinline|set!| will fail.

The second forms are equivalent to the following:
\begin{lstlisting}
(define symbol (lambda ([arg...]) code [code...]))
\end{lstlisting}

\subsubsection*{Examples}
\begin{lstlisting}
(define x 1)
(define value "value")
(define another-value (+ 1 2 3)) ; another-value is then equal to 6
(define (make-adder x) (lambda (n) (+ x n)))
\end{lstlisting}

	\subsection{Syntax lambda}
\label{builtins/lambda}

\subsubsection*{Declaration}
\begin{lstlisting}
(lambda ([arg...]) code [code...])
\end{lstlisting}

\subsubsection*{Parameters}
\begin{description}
	\item[arg...] Zero or more unevaluated symbols to bind values to when the given lambda
	\item[code/code...] An expression that will be evaluated when the produced lambda object is called. These code expressions are executed in the order they are given.
\end{description}

\subsubsection*{Description}
Creates a procedure object that can be used to execute user-defined code. Will return the evaluated result of the last code argument when executed.

A lambda needs at least one code element in its body.

\subsubsection*{Examples}
\begin{lstlisting}
; Returns the integer 42 when executed.
(lambda () 42)

; Binds the value 42 to the symbol x.
(define x ((lambda () 42)))

; Short-hand syntax to create lambda objects.
(define (make-adder x) (lambda (n) (+ x n)))
\end{lstlisting}

	\subsection{Syntax 'quote'}
\label{builtins/quote}

\subsubsection*{Declaration}
\begin{lstlisting}
(quote expression)
\end{lstlisting}

\subsubsection*{Parameters}
\begin{description}
	\item[expression] The expression is not evaluated but returned as the symbol or list that it represents.
\end{description}

\subsubsection*{Description}
Quotes the given expression argument. Note that only one argument can be specified.

\subsubsection*{Examples}
\begin{lstlisting}
(quote hello) ; Yields the symbol 'hello', not the value that might be bound to that symbol
(quote (1 2 3)) ; Logically equivalent to (list 1 2 3)
\end{lstlisting}

	\subsection{Syntax 'begin'}
\label{builtins/begin}

\subsubsection*{Declaration}
\begin{lstlisting}
(begin [expression...])
\end{lstlisting}

\subsubsection*{Parameters}
\begin{description}
	\item[expression...] One or more expressions that will be evaluated in the same order that they are given.
\end{description}

\subsubsection*{Description}
Acts basically like a \lstinline|lambda| except that it does not accept additional arguments and does not create an inner environment. The surrounding environment becomes the one within the \lstinline|begin| statement. When used in an assignment or return context, the last evaluated result within the \lstinline|begin| statement's body will be returned.

\subsubsection*{Examples}
\begin{lstlisting}
(begin (print "hello") (print "world")) ; Prints "hello\nworld\n"
(define x (begin 1 2 3)) ; x equals to 3 now.

; Will never return 1337 but print a string and return 42.
(if #f 1337 (begin (print "Returning the answer.") 42))
\end{lstlisting}

	\subsection{Syntax 'if'}
\label{builtins/if}

\subsubsection*{Declaration}
\begin{lstlisting}
(if condition then else)
\end{lstlisting}

\subsubsection*{Parameters}
\begin{description}
	\item[condition] The condition to be evaluated.
	\item[then] Will be evaluated if the given condition is not \lstinline|#f|.
	\item[else] Will be evaluated if the given condition is \lstinline|#f|.
\end{description}

\subsubsection*{Description}
If the given \lstinline|condition| evaluates to something that is not \lstinline|#f|, the \lstinline|then| argument is evaluated and the result of that is returned. In case the given \lstinline|condition| is \lstinline|#f|, the \lstinline|else| argument is evaluated and returned.

Note that it is guaranteed that either \lstinline|then| or \lstinline|else| is evaluated, but never both or none.

\subsubsection*{Examples}
\begin{lstlisting}
(if #t 1 2) ; Yields 1
(if #f 1 2) ; Yields 2
(if #f (print "Hello") (print "Goodbye")) ; prints "Goodbye"
\end{lstlisting}

	\subsection{Syntax 'and' and 'or'}
\label{<key>}

\subsubsection*{Declaration}
\begin{lstlisting}
(and [expression...])
(or [expression...])
\end{lstlisting}

\subsubsection*{Parameters}
\begin{description}
	\item[expression...] An expression to be evaluated. Note that all expressions are evaluated lazily.
\end{description}

\subsubsection*{Description}
Syntax \lstinline|and| evaluates the given expressions in the order they are given and returns \true{} if \textbf{all} evaluate to be not \false{}. The first expression that is evaluated to be \false{} will cancel the evaluation of the remaining expressions. If no argument is given, \true{} is returned.

Syntax \lstinline|or| evaluates the given expressions in the order they are given and returns \true if \textbf{any} evaluate to be \true{}. The first expression that is evaluated to be \true{} will cancel the evaluation of the remaining expressions. If no argument is given, \false{} is returned.

\subsubsection*{Examples}
\begin{lstlisting}
(and )
; Both examples below will always print "hello"
(and (print "hello") #f (print "world"))
(or  #f (print "hello") (print "world"))
\end{lstlisting}

	\subsection{Syntax 'time'}
\label{builtins/time}

\subsubsection*{Declaration}
\begin{lstlisting}
(time expression)
\end{lstlisting}

\subsubsection*{Parameters}
\begin{description}
	\item[expression] The expression to time.
\end{description}

\subsubsection*{Description}
Measures the time it takes to evaluate the given expression and returns an object containing this timing information.

\subsubsection*{Examples}
\begin{lstlisting}
(time 1)
(time (fac 20))
(time (tak 3 6 12))
\end{lstlisting}

	\subsection{Syntax 'assert'}
\label{builtins/assert}

\subsubsection*{Declaration}
\begin{lstlisting}
(assert condition [message])
\end{lstlisting}

\subsubsection*{Parameters}
\begin{description}
	\item[condition] The condition to be tested.
	\item[message] An optional message of type string which will be output in case the condition evaluates to \false{}.
\end{description}

\subsubsection*{Description}
Evaluates the given \lstinline|condition| and triggers an error when it is \false{}. If the \lstinline|message| argument is given, and it is a string, it is displayed as part of the error message. If the given condition evaluates to \true{}, nothing else is done.

\subsubsection*{Examples}
\begin{lstlisting}
(assert (and #t 42))
(assert #f "This condition always triggers an error.")
\end{lstlisting}


	% Procedures
	\subsection{Procedure '+', '-', '*', '/', and '\%'}
\label{builtins/arithmetic-operations}

\subsubsection*{Declaration}
\begin{lstlisting}
(+ [op...])
(- op [op...])
(* [op...])
(/ op [op...])
(% op op)
\end{lstlisting}

\subsubsection*{Parameters}
\begin{description}
	\item[op] An operand of the arithmetic operation. Must be of either integer or a float type. Note that the procedure '\%' only accepts integer type operands.
\end{description}

\subsubsection*{Description}
These are the typical arithmetic operations one would expect. The procedures \lstinline|+| and \lstinline|*| return 0 and 1 when no argument is given, respectively. The procedure \lstinline|%| requires exactly 2 arguments of type integer, all other arithmetic operations take either integer or float type numbers, or a mix of them.

\subsubsection*{Examples}
\begin{lstlisting}
(+ 1 2) ; Yields 3
(+ -1 2) ; Yields 1
(- 44 2) ; Yields 42
(*) ; Yields 1
(* 7 6) ; Yields 42
(/ 1 2) ; Yields 0.5
(% 29 12) ; Yields 5
\end{lstlisting}

	\subsection{Procedure '>', '>=', '=', '<', and '<='}
\label{builtins/arithmetic-comparison}

\subsubsection*{Declaration}
\begin{lstlisting}
(>  op op [op...])
(>= op op [op...])
(=  op op [op...])
(<  op op [op...])
(<= op op [op...])
\end{lstlisting}

\subsubsection*{Parameters}
\begin{description}
	\item[op] The operand of either integer or float type to compare against one or more other operands.
\end{description}

\subsubsection*{Description}
These are the typical arithmetic comparison operators. Each of them takes at least 2 arguments of either integer or float type.

\subsubsection*{Examples}
\begin{lstlisting}
(<  1 2) ; #t
(<= 1 2) ; #t
(=  1 2) ; #f
(>  1 2) ; #f
(>= 1 2) ; #f
\end{lstlisting}

	\subsection{Procedure 'read'}
\label{builtins/read}

\subsubsection*{Declaration}
\begin{lstlisting}
(read the-string)
\end{lstlisting}

\subsubsection*{Parameters}
\begin{description}
	\item[the-string] The string type object to be parsed.
\end{description}

\subsubsection*{Description}
Parses a given string for \lisp{} syntax and returns an object that can be evaluated by the function \lstinline|eval| (see \ref{builtins/eval}).

\subsubsection*{Examples}
\begin{lstlisting}
(read "1") ; Returns the integer 1
(read "hello") ; Returns the symbol 'hello'

; Returns a cons list containing 2 symbols and 1 integer.
; It is logically equivalent to (quote (define x 1))
(read "(define x 1)")
\end{lstlisting}

	\subsection{Procedure 'eval'}
\label{builtins/eval}

\subsubsection*{Declaration}
\begin{lstlisting}
(eval object)
\end{lstlisting}

\subsubsection*{Parameters}
\begin{description}
	\item[object] Any type of object that will be evaluated.
\end{description}

\subsubsection*{Description}
Returns the result of evaluating the given \lstinline|object|.

\subsubsection*{Examples}
\begin{lstlisting}[escapechar=`]
(eval 1) ; Evaluates to itself
(eval "foo") ; Also evaluates to itself

; Evaluates the result of read (see `\ref{builtins/read}`)
(eval (read "1")) ; Evalutes to the integer 1
\end{lstlisting}

	\subsection{Procedure 'print'}
\label{builtins/print}

\subsubsection*{Declaration}
\begin{lstlisting}
(print object)
\end{lstlisting}

\subsubsection*{Parameters}
\begin{description}
	\item[object] The object to print. May be of any type.
\end{description}

\subsubsection*{Description}
Prints the given object to the standard output.

\subsubsection*{Examples}
\begin{lstlisting}
(print 1)
(print "hello world")
(print (lambda() 42))
\end{lstlisting}

	\subsection{Procedure 'exit'}
\label{builtins/exit}

\subsubsection*{Declaration}
\begin{lstlisting}
(exit [code])
\end{lstlisting}

\subsubsection*{Parameters}
\begin{description}
	\item[code] An integer type object containing the exit code. If it is not given, the exit code 0 is used by default.
\end{description}

\subsubsection*{Description}
Exits the interpreter session with the optional exit code or 0.

\subsubsection*{Examples}
\begin{lstlisting}
(exit)
(exit 42)
\end{lstlisting}

	\subsection{Procedure 'cons'}
\label{builtins/cons}

\subsubsection*{Declaration}
\begin{lstlisting}
(cons left right)
\end{lstlisting}

\subsubsection*{Parameters}
\begin{description}
	\item[left] An object of any type to be placed in the left part (car) of the cons cell.
	\item[right] An object of any type to be placed in the right part (cdr) of the cons cell.
\end{description}

\subsubsection*{Description}
Creates a new cons-cell that stores two \lisp{} objects.

\subsubsection*{Examples}
\begin{lstlisting}
(cons 1 2); Prints (1 . 2)
(cons 1 (cons 2 (cons 3 null))) ; Prints (1 2 3)
(cons (cons (cons 1 null) 2 ) 3) ; Prints (((1) . 2) . 3)
\end{lstlisting}

	\subsection{Procedure 'car' and 'cdr'}
\label{builtins/car-cdr}

\subsubsection*{Declaration}
\begin{lstlisting}
(car cons-cell)
(cdr cons-cell)
\end{lstlisting}

\subsubsection*{Parameters}
\begin{description}
	\item[cons-cell] An object of type cons.
\end{description}

\subsubsection*{Description}
The procedure \lstinline|car| returns the left (or first) object in the given cons cell. The procedure \lstinline|cdr| returns the right (or second) object in the given cons cell.

\subsubsection*{Examples}
\begin{lstlisting}
(car (cons 1 2)) ; Yields 1
(cdr (cons 1 2)) ; Yields 2
\end{lstlisting}

	\subsection{Procedure 'list'}
\label{builtins/list}

\subsubsection*{Declaration}
\begin{lstlisting}
(list [object...])
\end{lstlisting}

\subsubsection*{Parameters}
\begin{description}
	\item[object] May be of any type.
\end{description}

\subsubsection*{Description}
Creates a regular list of cons cells from the given arguments. If no arguments are given, the empty list \lstinline|null| (see \ref{builtins/symbols}) is returned.

\subsubsection*{Examples}
\begin{lstlisting}[escapechar=`]
(list) ; Yields the empty list 'null'

(list 1 2 3) ; Logically equivalent to (cons 1(cons 2(cons 3 null)))
\end{lstlisting}

	\subsection{Procedure 'eq?'}
\label{builtins/eq}

\subsubsection*{Declaration}
\begin{lstlisting}
(eq? object object)
\end{lstlisting}

\subsubsection*{Parameters}
\begin{description}
	\item[object] An object of any type.
\end{description}

\subsubsection*{Description}
Returns \true{} if both arguments are the exact same objects, else \false{} is returned.

\subsubsection*{Examples}
\begin{lstlisting}
(eq? (quote a) (quote a)) ; Yields #t
(eq? (cons 1 2) (cons 1 2)) ; Yields #f
\end{lstlisting}

	\subsection{Procedure 'get-recursion-limit' and 'set-recursion-limit'}
\label{builtins/recursion-limit}

\subsubsection*{Declaration}
\begin{lstlisting}
(set-recursion-limit limit)
(get-recursion-limit)
\end{lstlisting}

\subsubsection*{Parameters}
\begin{description}
	\item[limit] The new limit.
\end{description}

\subsubsection*{Description}
Gets or sets the current recursion limit. The default value is 255.

\subsubsection*{Examples}
\begin{lstlisting}
(set-recursion-limit 1000)
(get-recursion-limit) ; Yields 1000
\end{lstlisting}

	\subsection{Procedure 'file.open'}
\label{builtins/file.open}

\subsubsection*{Declaration}
\begin{lstlisting}
(file.open file-or-filename [file-mode])
\end{lstlisting}

\subsubsection*{Parameters}
\begin{description}
	\item[file-or-filename] Can be a string type object containing the name of the file to open or a file type object which will be re-open.
	\item[file-mode] A string object describing the mode in which to open the file. One of \lstinline|"r"|, \lstinline|"w"|, and \lstinline|"rw"|. Defaults to \lstinline|"r"|.
\end{description}

\subsubsection*{Description}
Opens a file with the given file mode and returns a handle to it.

\subsubsection*{Examples}
\begin{lstlisting}
(file.open "the-file.txt") ; Opens the-file.txt in read-only mode.
(file.open "the-file.txt" "rw") ; Opens the-file.txt in read-write mode.
\end{lstlisting}

	\subsection{Procedure 'file.is-open'}
\label{builtins/file.is-open}

\subsubsection*{Declaration}
\begin{lstlisting}
(file.is-open file-object)
\end{lstlisting}

\subsubsection*{Parameters}
\begin{description}
	\item[file-object] An object of type file.
\end{description}

\subsubsection*{Description}
Checks whether the given file object is open or not and returns \true{} or \false{} accordingly.

\subsubsection*{Examples}
\begin{lstlisting}
; Yields #t (if my-file.txt exists, of course...)
(file.is-open file.open("my-file.txt"))
\end{lstlisting}

	\subsection{Procedure 'file.close'}
\label{builtins/file.close}

\subsubsection*{Declaration}
\begin{lstlisting}
(file.close file-object)
\end{lstlisting}

\subsubsection*{Parameters}
\begin{description}
	\item[file-object] The handle to the file that should be closed.
\end{description}

\subsubsection*{Description}
Closes the given file. If the file was already closed, nothing is done.

\subsubsection*{Examples}
\begin{lstlisting}
; Works only if the-file.txt exists, of course.
(file.close (file.open "the-file.txt"))
\end{lstlisting}

	\subsection{Procedure 'file.read-string'}
\label{builtins/file.read-string}

\subsubsection*{Declaration}
\begin{lstlisting}
(file.read-string filename)
\end{lstlisting}

\subsubsection*{Parameters}
\begin{description}
	\item[filename] The name of the file to read the content from. Must be of type string.
\end{description}

\subsubsection*{Description}
Reads the content of the given file as a string and returns that. This procedure will take care of opening and closing the file properly, even in case of an error.

\subsubsection*{Examples}
\begin{lstlisting}
; Yields the file content as a string object
(file.read-string "my-file.txt")
\end{lstlisting}


\end{document}
