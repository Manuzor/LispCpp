% !TeX encoding = UTF-8
% !TeX spellcheck = en_US
% !TeX root = documentation.tex
\documentclass[a4paper]{scrartcl}
\usepackage{fontspec}
\setmonofont{Consolas}
\usepackage{color}
	\definecolor{codeComment}{RGB}{0,128,0}
	\definecolor{codeKeyword}{RGB}{0,0,255}
	\definecolor{codeIdentifier}{RGB}{64,64,64}
	\definecolor{codeString}{RGB}{163,21,21}
	\definecolor{codeLineNumbers}{RGB}{0,0,0}
\usepackage{listings}
	\lstset{language=Lisp}
	\lstset{basicstyle=\ttfamily\small}
	\lstset{keywordstyle=\color{codeKeyword}}
	\lstset{commentstyle=\color{codeComment}}
	\lstset{identifierstyle=\color{codeIdentifier}}
	\lstset{stringstyle=\color{codeString}}
	\lstset{showstringspaces=false}
	\lstset{tabsize=4}
	\lstset{captionpos=b}
	\lstset{numbers=none}
	\lstset{numberstyle=\ttfamily\color{codeLineNumbers}}
	\lstset{breaklines=true}
	\lstset{morekeywords={define,if,lambda,quote,begin}}
\usepackage{todonotes}
\usepackage{enumitem}
\setlist[description]{leftmargin=\parindent,labelindent=\parindent}
\usepackage[pdfborder={0 0 0 0}]{hyperref}

\newcommand{\false}{\lstinline|\#f|}
\newcommand{\true}{\lstinline|\#t|}

% Title Page
\title{LispCpp\\
	Documentation and Reference Manual}
\subtitle{For the Course\\
	"Design and Implementation of Modern Programming languages"\\
	in Summer Term 2014 Held by Claus Gittinger}
\author{Manuel Maier, Matriculation Number 28535\\
		Stuttgart Media University}
\date{\today}

\begin{document}
\maketitle
\tableofcontents
\clearpage

\section{Features}
	\todo{Describe all features in a list with concise comments.}
	
\section{Built-in Symbols, Syntax, and Procedures}
	Within this section, a simple declaration syntax is used, mixed with the regular LISP syntax:
	
	\begin{itemize}
		\item Parameters preceded by an ellipse describe that the parameter may appear zero, one, or multiple times: \lstinline|param...|
		\item Parameters enclosed in brackets are optional: \lstinline|[param]| or \lstinline|[param...]|
		\item Procedure declarations follow this pattern:\\
			  \lstinline|(procedure-name required [optional])|
	\end{itemize}

	Everything else is regular LISP syntax. Note that these are not the actual declarations, they're just there for the purposes of this documentation. These are built-ins after all.

	\subsection{Pre-defined symbols}
\label{builtins/symbols}

\subsubsection*{Symbols}
\begin{lstlisting}
#t #f #v null
\end{lstlisting}

\subsubsection*{Description}
The symbol \lstinline|#t| represents the value 'true'.\\
The symbol \lstinline|#f| represents the value 'false'.\\
The symbol \lstinline|#v| represents the non-value, i.e. "void".\\
The symbol \lstinline|null| represents the empty list.

\subsubsection*{Examples}
\begin{lstlisting}
(if #t 1 2) ; Will always yield 1
(if #f 1 2) ; Will always yield 2
(print #v) ; Will never print anything
(cons 1 (cons 2 null)) ; Creates a regular list equivalent to '(1 2)
\end{lstlisting}


	\subsection{Syntax 'define' and 'set!'}
\label{builtins/define}

\subsubsection*{Declaration}
\begin{lstlisting}
(define symbol expression)
(define (symbol [arg]) code [code...])
(set! symbol expression)
(set! (symbol [arg]) code [code...])
\end{lstlisting}

\subsubsection*{Parameters}
\begin{description}
	\item[symbol] The unevaluated symbol to bind the result of the expression to.
	\item[expression] The result of expression is bound to the specified symbol.
	\item[code] An expression that will be evaluated when the lambda is called.
	\item[arg] An optional unevaluated symbol to bind values to that are available when executing all code argument.
\end{description}

\subsubsection*{Description}
Binds the value of the given evaluated expression to the given unevaluated symbol.

The only difference between \lstinline|define| and \lstinline|set!| is basically that \lstinline|set!| can only overwrite existing bindings. If a binding for a symbol does not exist yet, \lstinline|set!| will fail.

The second forms are equivalent to the following:
\begin{lstlisting}
(define symbol (lambda ([arg...]) code [code...]))
\end{lstlisting}

\subsubsection*{Examples}
\begin{lstlisting}
(define x 1)
(define value "value")
(define another-value (+ 1 2 3)) ; another-value is then equal to 6
(define (make-adder x) (lambda (n) (+ x n)))
\end{lstlisting}

	\subsection{Syntax lambda}
\label{builtins/lambda}

\subsubsection*{Declaration}
\begin{lstlisting}
(lambda ([arg...]) code [code...])
\end{lstlisting}

\subsubsection*{Parameters}
\begin{description}
	\item[arg...] Zero or more unevaluated symbols to bind values to when the given lambda
	\item[code/code...] An expression that will be evaluated when the produced lambda object is called. These code expressions are executed in the order they are given.
\end{description}

\subsubsection*{Description}
Creates a procedure object that can be used to execute user-defined code. Will return the evaluated result of the last code argument when executed.

A lambda needs at least one code element in its body.

\subsubsection*{Examples}
\begin{lstlisting}
; Returns the integer 42 when executed.
(lambda () 42)

; Binds the value 42 to the symbol x.
(define x ((lambda () 42)))

; Short-hand syntax to create lambda objects.
(define (make-adder x) (lambda (n) (+ x n)))
\end{lstlisting}

	\subsection{Syntax 'quote'}
\label{builtins/quote}

\subsubsection*{Declaration}
\begin{lstlisting}
(quote expression)
\end{lstlisting}

\subsubsection*{Parameters}
\begin{description}
	\item[expression] The expression is not evaluated but returned as the symbol or list that it represents.
\end{description}

\subsubsection*{Description}
Quotes the given expression argument. Note that only one argument can be specified.

\subsubsection*{Examples}
\begin{lstlisting}
(quote hello) ; Yields the symbol 'hello', not the value that might be bound to that symbol
(quote (1 2 3)) ; Logically equivalent to (list 1 2 3)
\end{lstlisting}

	\subsection{Syntax 'begin'}
\label{builtins/begin}

\subsubsection*{Declaration}
\begin{lstlisting}
(begin [expression...])
\end{lstlisting}

\subsubsection*{Parameters}
\begin{description}
	\item[expression...] One or more expressions that will be evaluated in the same order that they are given.
\end{description}

\subsubsection*{Description}
Acts basically like a \lstinline|lambda| except that it does not accept additional arguments and does not create an inner environment. The surrounding environment becomes the one within the \lstinline|begin| statement. When used in an assignment or return context, the last evaluated result within the \lstinline|begin| statement's body will be returned.

\subsubsection*{Examples}
\begin{lstlisting}
(begin (print "hello") (print "world")) ; Prints "hello\nworld\n"
(define x (begin 1 2 3)) ; x equals to 3 now.

; Will never return 1337 but print a string and return 42.
(if #f 1337 (begin (print "Returning the answer.") 42))
\end{lstlisting}

	\subsection{Syntax 'if'}
\label{builtins/if}

\subsubsection*{Declaration}
\begin{lstlisting}
(if condition then else)
\end{lstlisting}

\subsubsection*{Parameters}
\begin{description}
	\item[condition] The condition to be evaluated.
	\item[then] Will be evaluated if the given condition is not \lstinline|#f|.
	\item[else] Will be evaluated if the given condition is \lstinline|#f|.
\end{description}

\subsubsection*{Description}
If the given \lstinline|condition| evaluates to something that is not \lstinline|#f|, the \lstinline|then| argument is evaluated and the result of that is returned. In case the given \lstinline|condition| is \lstinline|#f|, the \lstinline|else| argument is evaluated and returned.

Note that it is guaranteed that either \lstinline|then| or \lstinline|else| is evaluated, but never both or none.

\subsubsection*{Examples}
\begin{lstlisting}
(if #t 1 2) ; Yields 1
(if #f 1 2) ; Yields 2
(if #f (print "Hello") (print "Goodbye")) ; prints "Goodbye"
\end{lstlisting}

	\subsection{Syntax 'and' and 'or'}
\label{<key>}

\subsubsection*{Declaration}
\begin{lstlisting}
(and [expression...])
(or [expression...])
\end{lstlisting}

\subsubsection*{Parameters}
\begin{description}
	\item[expression...] An expression to be evaluated. Note that all expressions are evaluated lazily.
\end{description}

\subsubsection*{Description}
Syntax \lstinline|and| evaluates the given expressions in the order they are given and returns \true{} if \textbf{all} evaluate to be not \false{}. The first expression that is evaluated to be \false{} will cancel the evaluation of the remaining expressions. If no argument is given, \true{} is returned.

Syntax \lstinline|or| evaluates the given expressions in the order they are given and returns \true if \textbf{any} evaluate to be \true{}. The first expression that is evaluated to be \true{} will cancel the evaluation of the remaining expressions. If no argument is given, \false{} is returned.

\subsubsection*{Examples}
\begin{lstlisting}
(and )
; Both examples below will always print "hello"
(and (print "hello") #f (print "world"))
(or  #f (print "hello") (print "world"))
\end{lstlisting}

	\subsection{Syntax 'time'}
\label{builtins/time}

\subsubsection*{Declaration}
\begin{lstlisting}
(time expression)
\end{lstlisting}

\subsubsection*{Parameters}
\begin{description}
	\item[expression] The expression to time.
\end{description}

\subsubsection*{Description}
Measures the time it takes to evaluate the given expression and returns an object containing this timing information.

\subsubsection*{Examples}
\begin{lstlisting}
(time 1)
(time (fac 20))
(time (tak 3 6 12))
\end{lstlisting}

	\subsection{Syntax 'assert'}
\label{builtins/assert}

\subsubsection*{Declaration}
\begin{lstlisting}
(assert condition [message])
\end{lstlisting}

\subsubsection*{Parameters}
\begin{description}
	\item[condition] The condition to be tested.
	\item[message] An optional message of type string which will be output in case the condition evaluates to \false{}.
\end{description}

\subsubsection*{Description}
Evaluates the given \lstinline|condition| and triggers an error when it is \false{}. If the \lstinline|message| argument is given, and it is a string, it is displayed as part of the error message. If the given condition evaluates to \true{}, nothing else is done.

\subsubsection*{Examples}
\begin{lstlisting}
(assert (and #t 42))
(assert #f "This condition always triggers an error.")
\end{lstlisting}


	\subsection{Procedure '+', '-', '*', '/', and '\%'}
\label{builtins/arithmetic-operations}

\subsubsection*{Declaration}
\begin{lstlisting}
(+ [op...])
(- op [op...])
(* [op...])
(/ op [op...])
(% op op)
\end{lstlisting}

\subsubsection*{Parameters}
\begin{description}
	\item[op] An operand of the arithmetic operation. Must be of either integer or a float type. Note that the procedure '\%' only accepts integer type operands.
\end{description}

\subsubsection*{Description}
These are the typical arithmetic operations one would expect. The procedures \lstinline|+| and \lstinline|*| return 0 and 1 when no argument is given, respectively. The procedure \lstinline|%| requires exactly 2 arguments of type integer, all other arithmetic operations take either integer or float type numbers, or a mix of them.

\subsubsection*{Examples}
\begin{lstlisting}
(+ 1 2) ; Yields 3
(+ -1 2) ; Yields 1
(- 44 2) ; Yields 42
(*) ; Yields 1
(* 7 6) ; Yields 42
(/ 1 2) ; Yields 0.5
(% 29 12) ; Yields 5
\end{lstlisting}

	\subsection{Procedure '>', '>=', '=', '<', and '<='}
\label{builtins/arithmetic-comparison}

\subsubsection*{Declaration}
\begin{lstlisting}
(>  op op [op...])
(>= op op [op...])
(=  op op [op...])
(<  op op [op...])
(<= op op [op...])
\end{lstlisting}

\subsubsection*{Parameters}
\begin{description}
	\item[op] The operand of either integer or float type to compare against one or more other operands.
\end{description}

\subsubsection*{Description}
These are the typical arithmetic comparison operators. Each of them takes at least 2 arguments of either integer or float type.

\subsubsection*{Examples}
\begin{lstlisting}
(<  1 2) ; #t
(<= 1 2) ; #t
(=  1 2) ; #f
(>  1 2) ; #f
(>= 1 2) ; #f
\end{lstlisting}

	\subsection{Procedure 'read'}
\label{builtins/read}

\subsubsection*{Declaration}
\begin{lstlisting}
(read the-string)
\end{lstlisting}

\subsubsection*{Parameters}
\begin{description}
	\item[the-string] The string type object to be parsed.
\end{description}

\subsubsection*{Description}
Parses a given string for \lisp{} syntax and returns an object that can be evaluated by the function \lstinline|eval| (see \ref{builtins/eval}).

\subsubsection*{Examples}
\begin{lstlisting}
(read "1") ; Returns the integer 1
(read "hello") ; Returns the symbol 'hello'

; Returns a cons list containing 2 symbols and 1 integer.
; It is logically equivalent to (quote (define x 1))
(read "(define x 1)")
\end{lstlisting}

	\subsection{Procedure 'eval'}
\label{builtins/eval}

\subsubsection*{Declaration}
\begin{lstlisting}
(eval object)
\end{lstlisting}

\subsubsection*{Parameters}
\begin{description}
	\item[object] Any type of object that will be evaluated.
\end{description}

\subsubsection*{Description}
Returns the result of evaluating the given \lstinline|object|.

\subsubsection*{Examples}
\begin{lstlisting}[escapechar=`]
(eval 1) ; Evaluates to itself
(eval "foo") ; Also evaluates to itself

; Evaluates the result of read (see `\ref{builtins/read}`)
(eval (read "1")) ; Evalutes to the integer 1
\end{lstlisting}

	\subsection{Procedure 'print'}
\label{builtins/print}

\subsubsection*{Declaration}
\begin{lstlisting}
(print object)
\end{lstlisting}

\subsubsection*{Parameters}
\begin{description}
	\item[object] The object to print. May be of any type.
\end{description}

\subsubsection*{Description}
Prints the given object to the standard output.

\subsubsection*{Examples}
\begin{lstlisting}
(print 1)
(print "hello world")
(print (lambda() 42))
\end{lstlisting}

	\subsection{Procedure 'exit'}
\label{builtins/exit}

\subsubsection*{Declaration}
\begin{lstlisting}
(exit [code])
\end{lstlisting}

\subsubsection*{Parameters}
\begin{description}
	\item[code] An integer type object containing the exit code. If it is not given, the exit code 0 is used by default.
\end{description}

\subsubsection*{Description}
Exits the interpreter session with the optional exit code or 0.

\subsubsection*{Examples}
\begin{lstlisting}
(exit)
(exit 42)
\end{lstlisting}

	\subsection{Procedure 'cons'}
\label{builtins/cons}

\subsubsection*{Declaration}
\begin{lstlisting}
(cons left right)
\end{lstlisting}

\subsubsection*{Parameters}
\begin{description}
	\item[left] An object of any type to be placed in the left part (car) of the cons cell.
	\item[right] An object of any type to be placed in the right part (cdr) of the cons cell.
\end{description}

\subsubsection*{Description}
Creates a new cons-cell that stores two \lisp{} objects.

\subsubsection*{Examples}
\begin{lstlisting}
(cons 1 2); Prints (1 . 2)
(cons 1 (cons 2 (cons 3 null))) ; Prints (1 2 3)
(cons (cons (cons 1 null) 2 ) 3) ; Prints (((1) . 2) . 3)
\end{lstlisting}

	\subsection{Procedure 'car' and 'cdr'}
\label{builtins/car-cdr}

\subsubsection*{Declaration}
\begin{lstlisting}
(car cons-cell)
(cdr cons-cell)
\end{lstlisting}

\subsubsection*{Parameters}
\begin{description}
	\item[cons-cell] An object of type cons.
\end{description}

\subsubsection*{Description}
The procedure \lstinline|car| returns the left (or first) object in the given cons cell. The procedure \lstinline|cdr| returns the right (or second) object in the given cons cell.

\subsubsection*{Examples}
\begin{lstlisting}
(car (cons 1 2)) ; Yields 1
(cdr (cons 1 2)) ; Yields 2
\end{lstlisting}

	\subsection{Procedure 'list'}
\label{builtins/list}

\subsubsection*{Declaration}
\begin{lstlisting}
(list [object...])
\end{lstlisting}

\subsubsection*{Parameters}
\begin{description}
	\item[object] May be of any type.
\end{description}

\subsubsection*{Description}
Creates a regular list of cons cells from the given arguments. If no arguments are given, the empty list \lstinline|null| (see \ref{builtins/symbols}) is returned.

\subsubsection*{Examples}
\begin{lstlisting}[escapechar=`]
(list) ; Yields the empty list 'null'

(list 1 2 3) ; Logically equivalent to (cons 1(cons 2(cons 3 null)))
\end{lstlisting}


\end{document}
